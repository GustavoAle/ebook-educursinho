
\setlength{\parskip}{1mm}
\fontsize{11}{13}
\selectfont
O projeto EduCursinho - Educação e Aprovação é uma atividade de extensão 
da Universidade Federal de Mato Grosso (UFMT) que tem por objetivo 
oferecer a estudantes em processo de preparação para prestarem o Exame 
Nacional do Ensino Médio (Enem) e que sejam, preferencialmente, 
pertencentes a famílias de baixa renda, uma série de atividades educativas 
(aulas, resolução de exercícios, dicas e macetes etc.) que os auxiliem na 
obtenção de êxito no acesso ao ensino superior.

Por entender que a Universidade Pública é parte do desenvolvimento de uma 
sociedade, as ações de extensão surgem como uma das formas de contato 
entre a comunidade externa e a acadêmica - estudantes e servidores – 
fazendo com que a Universidade cumpra o seu papel de atuação na sociedade 
de maneira ampla e geral. A Universidade se torna parte do convívio 
social, quando, por meio de suas ações, promove um retorno à população com 
atividades e propostas que são alcançáveis por um ou vários grupos sociais.

Desde a sua idealização, a proposta é motivar estudantes universitários a 
estabelecerem um diálogo com a sociedade externa à universidade promovendo 
ações que atendam diretamente às demandas sociais e que proporcionem a 
troca de conhecimentos entre os mesmos. Neste projeto, as ações iniciais 
são a formação e a capacitação dos estudantes da UFMT para atuarem como 
professores nas diversas áreas contempladas pelo Enem: Ciências Humanas e 
suas Tecnologias, Ciências da Natureza e suas Tecnologias, Linguagens, 
Códigos e suas Tecnologias e Matemática e suas Tecnologias.

Nesse sentido, o projeto possui a seguinte estruturação interna: 
coordenação geral, coordenadores de área (Física, Química, Matemática, 
Redação, Atualidades e Inglês) e coordenador de marketing, sendo cada uma 
dessas áreas gerida por um docente (interno ou externo) da UFMT. Além 
disso, há os estudantes que atuam como líderes de divulgação e organização 
e os responsáveis por cada grupo de conteúdo, subdivididos de acordo com 
as respectivas disciplinas.

Com o advento da pandemia do Coronavírus (Covid-19), as ações tiveram de 
ser alteradas, de modo a reduzir e/ou evitar prejuízos causados pelas 
medidas de distanciamento que foram necessárias para conter o avanço do 
vírus. Dessa forma, todo o projeto foi reestruturado para atender às 
demandas de ensino por meio da utilização de estratégias já conhecidas do 
ensino a distância. As aulas passaram a ser ministradas mediante 
ferramentas digitais, como aulas por meio de plataformas como Youtube, 
Google Classroom e Google Meet.

Diante do exposto, surgiu a ideia de elaborar uma coleção de livros 
digitais (e-books), que foram elaborados pela equipe do projeto (docentes/
estudantes) e divididos em cinco (5) volumes, de forma a tornar o acesso 
ao material de estudo por parte do público-alvo ainda mais fácil e 
interativo. Esses e-books foram publicados em parceria com a Editora da 
UFMT (EdUFMT) e são disponibilizados de forma totalmente gratuita aos 
alunos, reforçando ainda mais a missão da universidade de impactar 
positivamente a sociedade com ações que promovam crescimento social, tanto 
de forma local, como de forma ampla. 

Esperamos que essa coleção, por meio de cada um dos volumes, seja 
realmente aproveitada ao máximo, pois fizemos tudo com muito carinho e 
dedicação para que você tenha cada vez mais acesso a informações de 
qualidade. O seu potencial está aí dentro, então use as ferramentas 
disponíveis para alavancar isso e mostre que o caminho vai ser trilhado 
com muita diligência e perseverança. Bons estudos e se liga nas dicas que 
colocamos para vocês nos capítulos.\\

\noindent
Atenciosamente,

\hspace*{\fill} Equipe EduCursinho.


\vfill
\noindent
\textit{Coordenadora Geral: Daniele Caetano da Silva}\\

\noindent
\textit{Coordenador de Física: Murilo José Pereira de Macedo}\\
\textit{- Danilo Oliveira dos Reis Nascimento}\\
\textit{- Luana Andra Correa Teixeira}\\
\textit{- Yhan Toth}\\

\noindent
\textit{Coordenadora de Inglês: Mônica Aragona}\\
\textit{- Jessica de Almeida Barradas}\\   

\noindent
\textit{Coordenadora de Matemática: Gláucia Aparecida Soares}\\
\textit{- Gustavo Mene Ale Primo}\\
\textit{- Rafael Simões Martins da Silva Oliveira}\\
\textit{- Wesley Alexei Paiva}\\

\noindent
\textit{Coordenador de Marketing: Felipe Thomaz Aquino}\\
\textit{- Gustavo Mene Ale Primo}\\
\textit{- Livia Pereira de Alencar}\\
\textit{- Matheus Henrique Sampaio Costa e Silva}\\

\noindent
\textit{Coordenadora de Química: Daniele Caetano da Silva}\\
\textit{- Antônio Gierdson Lima dos Santos}\\
\textit{- Jeferson Mangueira de Castro Lydijusse}\\
\textit{- Stefanie Santos Silva}\\

\noindent
\textit{Coordenadora da Redação/Atualidades: Manuella Soares Jovem}\\
\textit{- Nicole Astutti Campos}\\
\textit{- Diane Santana Brito}\\
\textit{- João Pedro dos Santos Menezes}\\



\setlength{\parskip}{0mm}